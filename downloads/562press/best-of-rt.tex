% the following vim mapping was helpful for creating new items:
% map <Leader>p i\paragraph{\#.} \intro{<+++>}: <+++><ESC><ESC>T#i

\documentclass[10pt]{memoir}
\setstocksize{8.5in}{5.5in}
\settrimmedsize{\stockheight}{\stockwidth}{*}
\settypeblocksize{506pt}{361pt}{*}
\usepackage[top=1in, bottom=0.8in, left=0.8in, right=0.8in]{geometry}

\usepackage{ebgaramond}
\usepackage{microtype}

\usepackage{ifluatex,ifxetex}

% fix a bug in this font
\ifluatex\else\ifxetex\else
  \normalfont
  \makeatletter
  \input{TS1EBGaramond-LF.fd}
  \input{TS1EBGaramond-OsF.fd}
  \makeatother
\fi\fi

\usepackage{url}
\usepackage{hyperref}
\usepackage{graphicx}
\usepackage{tabularx}

% remove boxes around TOC entries and other hyperlinks
\hypersetup{
    colorlinks,
    citecolor=black,
    filecolor=black,
    linkcolor=black,
    urlcolor=black
}

% hang the numbers / paragraph headings into the margine
\newcommand{\marginbox}[1]{%
  \parbox[t][0pt]{6em}{\raggedleft\footnotesize \mbox{} #1 \thinspace}}
\newcommand{\marginhead}[1]{%
{\llap{\marginbox{#1}\kern0.5em}}}
\setparaindent{0em}
\setafterparaskip{0em}
\setparaheadstyle{\marginhead}
\setparahook{\setsecnumformat{\csname the##1\endcsname\ }}


% headers
\copypagestyle{ankibook}{plain}
\makeevenhead{ankibook}{\thepage}{\itshape The Best of Random Thoughts}{}
\makeoddhead{ankibook}{}{\leftmark}{\thepage}
\makeevenfoot{ankibook}{}{}{}
\makeoddfoot{ankibook}{}{}{}

% an ugly hack to allow formatting instead of using \leftmark
\makeatletter
\makepsmarks{ankibook}{
  \def\chaptermark##1{\markboth{%
    \ifnum \value{secnumdepth} < -1
      \if@mainmatter
        \chaptername\ \thechapter\ --- %
      \fi
    \fi
    ##1}{}}

  \def\sectionmark##1{\markright{%
    \ifnum \value{secnumdepth} < 0
      \thesection. \ %
    \fi
    ##1}}
}
\makeatother

% templates
\newcommand{\speakertag}[1]{\emph{#1}: }
\newcommand{\st}{\speakertag}
\newcommand{\intro}[1]{\emph{#1}}

% we use tabular for dialogues, not tables, so sep can be nice and small
\renewcommand{\tabcolsep}{2pt}

%\frenchspacing
\renewcommand{\contentsname}{\centerline{Contents}}

\begin{document}
%% Page 1: Title %%
\thispagestyle{empty}

\vspace*{8em}
\begin{center}
\hrule\vskip 5pt
{\HUGE
The Best of
\vskip 10pt

Random Thoughts
}
\vskip 4pt
\hrule
\vskip 15pt

{\LARGE \oldstylenums{2009}\thinspace--\thinspace\oldstylenums{2014}}
\end{center}

\vspace*{3em}

{\itshape
  \setlength\leftskip{0pt}
  \setlength\rightskip{0pt}
  \setlength\parfillskip{0pt}
  \setlength\parindent{0pt}

  A chaotic collection of quotations, anecdotes, incidents, complaints, ideas, thoughts, and just about everything else that anybody would think to record
}

\vspace{16em}

\begin{center}
{\Large Compiled by}
\vskip 5pt
{\Huge \textsw{Soren Bjornstad}}
\vskip 2pt
\end{center}

\clearpage

%% Page 2: Dedication & Introduction %%
\thispagestyle{empty}

\vspace*{9em}
  {
    \noindent Random Thoughts is copyright 2009--2014 Soren I. Bjornstad.

    \vspace*{1em}

    \noindent The portion of Random Thoughts that is included in this volume, as well as this formatted volume itself, is licensed under a CC-BY \liningnums{4.0} Attribution license. This means that you can share and adapt it, including for commercial uses, without asking my permission, so long as you provide appropriate attribution.

    \vspace*{0.5em}

    \begin{center}
    \includegraphics[scale=0.75]{ccby.png}

    \url{http://creativecommons.org/licenses/by/4.0/}
    \end{center}
  }

  \vspace{0.5em}

  \noindent This book was produced using \LaTeX, a free typesetting system. The text font is 10-point Garamond.

  \vfill
\begin{center}
  {\scshape
    five hundred sixty-two press, mmxiv\\
  }
\end{center}

\clearpage

%% Page 3: Table of Contents %%
{
  \newgeometry{left=1.5in, right=1.5in}

  \pagestyle{empty}
  \tableofcontents*
  \thispagestyle{empty} % yes, this has to come *after* to work
  \clearpage
}

%% Page 4+: Begin chapter/automatic pagination %%
\pagestyle{ankibook}
\chapterstyle{bianchi}
\openany
\aliaspagestyle{chapter}{ankibook} % use same page style for chapter openings as body pages

%\raggedright

\chapter{2009}
\paragraph{54} This is a little routine I made up, inspired by \emph{Who's On First} and based on stories told by the musical group \emph{Lost and Found}, which has actually published \textsc{cd}s titled \emph{This}, \emph{Something}, and \emph{Here}. A customer is attempting to purchase one of said \textsc{cd}s from somewhere (the precise location doesn't matter).
\bigskip

--- Hey, what \textsc{cd}s do you have here?

--- Yes.

--- What?

--- No. No \emph{What}.

--- No, that's what.

--- No, \emph{What} really isn't \emph{No}. We don't even have \emph{What}.

--- What!?

--- For the last time, we don't have \emph{What}.

--- Okay, I just want a \textsc{cd}!

--- All right. Do you want \emph{Something}?

--- Yes, of course, I said I wanted a \textsc{cd}! Which ones do you have?

--- \emph{Something}.

--- I see you have something. I want to know \emph{which} \textsc{cd} you have.

--- \emph{Something}.

--- Okay, let me see that.

--- We don't have \emph{That}.

--- What do you mean, you don't have that? You're holding it!

--- No, I'm holding \emph{Something}.

--- Come on! \emph{(grabs \textsc{cd})} Let me see this!

--- If you wanted to see \emph{This}, why did you take \emph{Something}?

--- I took this because I wanted to see it!

--- No, you didn't. \emph{This} is still over \emph{Here}. You took \emph{Something}.

--- I can see that that is over there. What is going on here?

--- \emph{That} isn't over there. We don't have a \emph{That}. \emph{This} is on top of \emph{Here},

% this is a hack, but there are only a couple of lines that run over and it's easier than setting the whole thing tabular
% in case you didn't figure it out, this manually adds the indent plus one em, which is (obviously) the length of the em-dash, plus a space, which lines it up perfectly
\noindent \hspace{\parindent}\hspace{1em} not a \emph{What}.

--- Can I see something else now?

--- Sure. Look at \emph{This}.

--- You aren't giving me anything.

--- Of course I'm not giving you anything. You asked for \emph{This}. \emph{This} is

\noindent \hspace{\parindent}\hspace{1em} over there.

--- Can I just buy something, please?

--- Okay, give \emph{Something} to me so I can check the price.

--- What, you can't check the price of anything unless I give you

\noindent \hspace{\parindent}\hspace{1em} something first?

--- No. You have the only copy of \emph{Something}, after all.

--- All right. You know what, I'll just buy this. \emph{(hands \emph{Something} to}

\noindent \hspace{\parindent}\hspace{1em} \emph{cashier)}

--- That's \emph{Something}, not \emph{This}. Do you want \emph{This}, or do you want

\noindent \hspace{\parindent}\hspace{1em} \emph{Something}?

--- Both! I want this, which is something, after all.

--- No, \emph{This} really isn't \emph{Something}.

--- Just sell me something!

--- Okay. Ten dollars. \emph{(pays)}

--- Now can I have that, please?

--- You want \emph{That} and \emph{Something}?

--- No, I want the \textsc{cd} I just paid for.

--- Oh, you want \emph{Something}.

--- \emph{No}, not just anything. The \textsc{cd} I paid for.

--- You mean \emph{Something}.

--- Whatever.

\paragraph{144} I was in the library with Nichi one day. She was using the library catalog system and I was standing right next to her. She typed in the keyword ``file-sharing" and a book called \emph{Downloading Music} came up on the screen~-- still inside the catalog, mind you, with a big image and copy information next to it. At this moment, one of the know-nothing librarians came walking across the room towards us on the way to the front of the library. She stopped, looked over Nichi's shoulder, took a good ten seconds to scan the screen, then said, “You're not allowed to download music on these computers. These computers are for card catalog use only.” Then she walked away without another word. We had a hard time not laughing until she got around the corner.

\paragraph{160} Three Microsoft engineers and three Apple engineers are taking a train to a conference together. When they get to the station, they go up to the ticket window, and the Microsoft engineers buy a ticket each, but the Apple engineers buy only one ticket among the three of them. ``How are three people going to travel on one ticket?'' the Microsoft engineers ask.

“Watch and you'll see,” say the Apple engineers. So they get on the train, and the Apple engineers squeeze into a bathroom.

The train starts moving, and the conductor comes down the train, bangs on the door of the bathroom where the Apple engineers are hiding, and says, “Ticket, please!” One of the Apple engineers opens the door a crack and hands the conductor the ticket.

The Microsoft engineers are watching this and think it's a pretty good idea, so, as usual, they decide to copy it on the way back. However, this time, the Apple engineers skip buying tickets entirely. ``How are three people going to travel without a ticket?" ask the Microsoft engineers.

“Watch and you'll see,” say the Apple engineers. So they get on the train, the Microsoft engineers go into a bathroom, and the Apple engineers go into another bathroom further down the train.

Shortly after the train starts moving, one of the Apple engineers comes out of the bathroom, walks down the train, knocks on the door of the bathroom where the Microsoft engineers are hiding, and says, “Ticket, please!”

\chapter{2010}
\paragraph{225} The following conversation took place after I played my solo at \textsc{issma} last year.\bigskip

\noindent \begin{tabularx}{\textwidth}{r X}
\st{Judge} & ``Have you grown a lot lately?" \\
\st{Me} & ``\ldots I guess so?" \\
\st{Judge} & ``Well, that’s not a full-size violin, is it?" \\
\st{Me} & ``Umm, I’m\ldots pretty sure it is." \\
\st{Judge} & ``Well, I guess it must be a slightly smaller full-size violin than \mbox{normal}." \\
\end{tabularx}

\paragraph{227} ``Before you criticize someone, walk a mile in their shoes. That way, if they get angry, you'll be a mile away, \emph{and} you'll have their shoes."

\paragraph{297} We were taking a test on Edline (our online grading system) in the computer lab the other day. Our normal teacher was gone, so there was some substitute teacher supervising the test. I started up the computer, opened Edline, took the test, scrolled back up to the top and double-checked my answers, then hit submit. I went back to the home page of Edline, and, having nothing more to do, took my novel from my pile of books and started reading. After I had read about a page, the sub came up behind me (where I could barely even see him) and said loudly, in an I-can't-believe-you-would-try-to-pull-that-one tone, ``I'm going to have to ask you to put that away." I was surprised because I hadn't seen him coming as well as rather confused at the accusation. (As if I would be reading answers out of a novel!) I said, “Well, I'm done with the test.” He walked away immediately, saying, “Oh, are you?” I was actually quite annoyed that he didn't apologize.

\paragraph{323} \intro{Overheard on the bus}: ``You know, that triangle with the square in the corner. The \emph{a} squared plus \emph{b} squared equals \emph{c} squared thingy.''

\paragraph{338} \intro{Error received when starting my computer}: ``There is no disk in the drive. Please insert a disk into drive \textbackslash Device\textbackslash Harddisk4.''

\paragraph{358} \intro{Uncle Jeff, realizing why his keyboard was acting up}: ``My plate was sitting on it.''

\paragraph{363} \intro{A very practical problem that anybody might need to solve some day, from my algebra textbook}: ``A man is rowing upstream from point \textsc{a}. At point \textsc{b}, he loses his hat. Ten minutes later, he notices the loss of his hat and immediately turns around and rows back. Exactly at point \textsc{a}, he overtakes his hat. What is the speed of the current?" 

\paragraph{441} \intro{In a newsletter from my orthodontist}: ``Please note: Battery acid is listed below only for purposes of comparison, and should never be confused for any reason as a beverage.''

\paragraph{506} \intro{Herr Holt told us that he once had a student in his introductory class who advised the following}: ``When in doubt, umlaut!''

\paragraph{518} \intro{Tess, in my World History class}: ``Why can't everyone just get along and stop fighting? It would make this class a lot easier.''

\paragraph{590} \intro{In my driving textbook}: ``The steering wheel controls the direction of the front wheels. Turn right to go right; turn left to go left.''

\paragraph{676} \intro{Riley, a student in my algebra class who was somewhere between the class clown and dunce}: ``You always throw tricks out the window, Mrs.\ Fiegle-Hicks!''

\chapter{2011}
\paragraph{791} \intro{An acquaintance that I sat with on the bus for a year or so, explaining part of his rationale for the alphabet he had invented for his constructed language}: ``So the English language won't be able to sue me.''

\paragraph{818} \intro{Gregory, commenting on my upcoming performance}: ``When I grunted, you used more bow, and it was good, so just imagine me grunting.''

\paragraph{820} So we lost an orchestra folder somewhere. Mr.\ Kopf told us to look around the room for it, saying that sometimes folders tended to get lost behind the curtains. (Marissa: ``Yup, sometimes the air currents just pick up a folder, and it goes `whoosh!' and lands behind the curtain.") After a while, it became clear that Taylor, who is not particularly favorably known to other members of the orchestra, had either taken the folder for some reason (it was not hers) or found it lying around somewhere. In any case, instead of simply returning it to its slot in the orchestra room as any sensible person would do, she dropped it off in the main office of the school. It then disappeared for a few days, after which it turned up in the Lost and Found and finally made its way back to the orchestra room.

\paragraph{903} ``And lead us not into temptation, but deliver us some email.''

\paragraph{917} \intro{Sign posted at the end of the cafeteria line}: ``STOP TAKING POP-TARTS! WER$\;\!\!$'$\:\!$E WATCHING YOU!" \emph{(sic)}

\paragraph{1035} \intro{Mrs.\ Brady told us this story, which somehow came up during precalculus class}: She was going on a trip with some friends. They were taking a bus across the country. They left in the evening, so after an hour or two everyone went to sleep. In the middle of the night at about \oldstylenums{2} or \oldstylenums{3} \textsc{am}, she woke up, and immediately knew something felt really weird. After thinking for a moment and looking out the window, she realized that the bus driver had missed his exit and was backing up on the freeway in order to get to it.

\paragraph{1064} Had a driving lesson today. The other guy driving with me was a \oldstylenums{21}-year-old Ivy Tech student who had never driven a car before, not even around a parking lot. Now the first thing you have to do in every lesson is to turn out onto Campbell Street, which isn't exactly low-traffic. So out we turn (me having driven my hour already) and we wind up on the left side of the road facing oncoming traffic. We steer to correct it and hit the curb. Down Campbell Street we go, narrowly missing parked cars and bumping into the curb as we go. Nearly every turn goes wide, and at one point we got literally half the car up onto the sidewalk. The scariest part, which ironically wasn't his fault at all, was when a lady driving a large black \textsc{suv} turned right way wide and nearly hit us. It was only by the guy's reaction (admirable given that he had only been driving for about forty minutes) of stepping on the gas and moving into the middle of the intersection on a red light that we avoided an accident. The instructor said, ``That was the closest I've ever come to being hit without being hit."

When we got back, after the guy had left and I was still waiting to be picked up, the instructor dispensed the following quotes:
\begin{itemize}
  \setlength{\itemsep}{-3pt}
  \item ``One of the worst I've ever had."
  \item ``Really earning my pay this time."
  \item ``Longest hour of my life."
  \item ``Thank you for not screaming."
\end{itemize}

\paragraph{1284} ``An apostrophe is the difference between a business that knows its shit and a business that knows it's shit." ---\thinspace\emph{Will Rolls}

\paragraph{1291} Two guys walk into a bar. The first guy, thinking himself clever, asks for some ``$\textrm{H}_2\textrm{O}$." The second guy, trying to imitate his clever friend, says, ``I'll have some $\textrm{H}_2\textrm{O}$, too." 

The second guy died.\footnote{If it's been too long since your last chemistry class, $\textrm{H}_2\textrm{O}_2$ is hydrogen peroxide.}

\paragraph{1292}~\\
\noindent \begin{tabularx}{\textwidth}{r X}
\noindent \st{My Mother} & ``Usually when one wants to see something, one puts it on a clear surface." \\
\st{Me} & ``But\ldots I don't have any clear surfaces." \\
\st{Her} & ``My point exactly." \\
\end{tabularx}

\paragraph{1341} This may sound like a hoax, but it really happened to me today. I was sitting at lunch with Daniel. One of the new administrators comes along and greets us.\\

\noindent \begin{tabularx}{\textwidth}{r X}
\noindent \st{Him} & ``Can you two make sure you don't leave your trays and trash here today?"\\
\st{Us} & ``\ldots huh?"\\
\st{Him} & ``Well, yesterday you left your trays and trash here." \\
\st{Us} & \emph{(baffled, having done nothing of the kind)} ``\ldots That wasn't us."\\
\st{Him} & \emph{(with a triumphant grin on his face)} ``Oh yeah? Well, we have it on video!"
\end{tabularx}\\\smallskip

\noindent He shows us the ``video evidence," which is an inkjet printout on recycled copy paper depicting two blurry figures barely recognizable as the two of us getting up from the table with our books.\\

\noindent \begin{tabularx}{\textwidth}{r X}
\st{Him} & \emph{(to Daniel)} ``You didn't have your tray with you when you got up." (Because the picture was of us getting up at the end of the period, after having already put our trays away.)\\
\st{Me} & \emph{(so as not to have to argue that)} ``Well, I took it over for him." (So we didn't both have to spend a minute walking to the other end of the cafeteria.)\\
\end{tabularx}\\\smallskip

\noindent He ignores that completely and goes on to describe how Daniel had shoved his tray off the table onto the floor or something, which was complete nonsense. Eventually, after some more arguing\ldots\\

\noindent \begin{tabularx}{\textwidth}{r X}
\st{Him} & ``Let's just make sure that you take your trays today."\\
\st{Us} & ``\ldots Okay, sure!"
\end{tabularx}\\\smallskip

\noindent So when we finished eating today, we got up, held our trays up to our shoulders, walked over to the trashcan right in front of the camera, and dumped our trash in. Then I very deliberately took his tray (while still holding it up at shoulder height in front of the camera) and placed it on top of mine, loudly said, ``You're welcome!", and walked off to put the trays on the return belt.


\paragraph{1389} We're having brunch at Pikk's downtown, and my dad tries to order a drink.\\

\noindent \begin{tabularx}{\textwidth}{r X}
\st{Waitress} & ``I'm sorry, we don't have the ingredients for that right now." \\
\st{My Dad} & ``Okay, I'll have a ginger beer then." \\
\st{Waitress} & ``I'm sorry, we're out of ginger beer." \\
\st{My Dad} & ``\ldots Okay, I'll have a root beer then." \\
\end{tabularx}\\\smallskip

\noindent She heads back to the kitchen. She has to come back to confirm his order. Five minutes later, she's back again.\\

\noindent \begin{tabularx}{\textwidth}{r X}
\st{Waitress} & ``I'm sorry, we're all out of root beer." \\
\st{My Dad} & ``Okay, how about the limeade then?" \\
\st{Waitress} & ``Sorry, we're out of limeade too." \\
\st{My Dad} & ``How about a lemonade?" \\
\st{Waitress} & ``All right, we have that!" \\
\end{tabularx}

\paragraph{1391} \intro{In my high school's student handbook}: ``Once students enter the parking lot, the car is to be parked.''

\paragraph{1410} \intro{Mr.\ Hefner, on unit cancellation:} ``There is a God, and he uses units.''

\paragraph{1435} So a student in my English class who shall remain nameless was doing a vocabulary assignment with me, where we had to provide definitions for some less-common words that would be coming up in our next reading. The word we were looking up was \emph{inane}, and the dictionary definition was \emph{ridiculous}. The student took it to mean \emph{hilarious}, and off she was writing a sentence about ``the inane clown." I corrected her politely, saying that I knew the word and it definitely did not mean \emph{ridiculous} in the sense of \emph{humorous,} to which she said:

``Soren, if you were a teacher, nobody would like you. You're so picky!"

\paragraph{1442} \intro{A student in my German class, when she was asked to comment on our discussion topic and couldn't think of anything to say:} ``Ich finde Molestation schlecht."\thinspace\footnote{\emph{I think molestation is bad.}}

\paragraph{1449} Mrs.\ Stoltzfus was ranting on and on for about five minutes on how one group should have known for sure what a question on her last quiz was asking because of the placement of the modifier (they had argued that they should get points back because the question was ambiguous and they'd misunderstood it). Then she went on to tell us all about misplaced modifiers. Then she handed out a quiz, and the third question contained a misplaced modifier.

\paragraph{1461} \intro{Me, describing a form I had had to fill out to someone}: ``It was a pillable fee-dee-eff.''\thinspace\footnote{i.e., a \emph{fillable \textsc{pdf}}}

\paragraph{1476} \intro{A sixth-grader in the Science Olympiad event I was coaching}: ``You know what sucks? I can't email things to myself, because Macs can't intercept it.''

\chapter{2012}

\paragraph{1575} (The school district has recently made news for making some accounting mistakes that have caused a large amount of money to disappear without a trace. The principal is doing end-of-the-day announcements over the loudspeaker.)\\

\noindent \begin{tabularx}{\textwidth}{r X}
\st{Him} & ``Will the student with the Viking Nation receipt box please come to the office as soon as possible?"\\
\st{Me} & ``So \emph{that's} where our 3 million dollars went!"
\end{tabularx}

\paragraph{1590} \intro{Frau Houldieson, about some famous person we were discussing}: ``Sie ist im dritten Weltkrieg gestorben.''\thinspace\footnote{\emph{She died in World War 3.}}

\paragraph{1594} \intro{Mr.\ Kopf}: ``Fraudenscheude.''\thinspace\footnote{i.e., \emph{Schadenfreude.}}

\paragraph{1602} \intro{Mrs.\ Stoltzfus}: ``Do the means justify the ends?''

\paragraph{1606} \intro{Mr.\ Kopf}: ``It was at the Chapel of the Insurrection.''

\paragraph{1631} \intro{Gregory}: ``If you play on the fingerboard, I will come down to Indy and shoot you.''

\paragraph{1642} \intro{Answer on a physics quiz, read to the class by Mr.\ Hefner}: ``The light that comes off a smooth surface is called a sensual reflection.''\thinspace\footnote{\emph{Specular}.}

\paragraph{1645} \intro{Overheard on the bus}: ``I lost my phone and my contacts, so now I don't know who to call [to buy more drugs].''

\paragraph{1692} I have a group of friends that goes to the school's Media Center every morning to chat, finish up some homework quickly, and so on. We enjoy giving the links on the bookmarks bars of the public computers nearby whimsical names, changing them regularly. Which brought someone to have to ask one morning: ``Which one is Google again? `Crash Mainframe'?"

\paragraph{1704} On a tape from when I was about six:\\

\noindent \begin{tabularx}{\textwidth}{r X}
  \st{My Dad} & ``So\ldots you knew [your otters] even before you were born?"\\
  \st{Me} & ``Yes."\\
  \st{My Dad} & ``What place was that?"\\
  \st{Me} & \emph{(very matter-of-factly)} ``In Washington."
\end{tabularx}

\paragraph{1725} \intro{Mr.\ Kopf, to me, after asking several people if they'd seen that they were in a picture in the morning newspaper}: ``Did you see your elbow in the paper?''

\paragraph{1744} \intro{Searching the library catalog:} ``No matches for \emph{dvorak keyboard}. Did you mean \emph{dork keyboard}?''

\paragraph{1745} Right before the end of last school year, the orchestra was playing live music for the ``Taste of Valpo" event at the Expo Center in Valpo. Because of the layout of the space, we were sitting in reverse, where the violins were sitting on the side of the orchestra with the opening of the \textsc{u} to the left, rather than to the right as normal. I was sitting on the inside of the first stand in my row (the second one back). Tess was on the outside at first, but she had to leave about three-quarters of the way through to go somewhere else.

No sooner has she left when some random guy comes up to the orchestra area, pulls Tess's chair about three feet away from my stand, and \emph{sits down} in it. He remains sitting there~-- right next to me, as I'm playing~-- for the next fifteen minutes, eating and chatting with friends who come up and for some reason don't have the sense or the nerve to ask him what the heck he's doing. I have never seen any audience behavior quite so weird before or since.

\paragraph{1758}
We're in church at \textsc{lsm} and I'm trying to share the peace with my counselor. I hold out my hand.\\

\noindent \begin{tabularx}{\textwidth}{r X}
  \st{Him} & ``What do you want from me?"\\
  \st{Me} & ``\ldots Nothing\ldots I'm just trying to shake your hand\ldots because it's the peace\ldots"\\
  \st{Him} & \emph{(pause)} ``Oh." \emph{(hesitantly shakes my hand)}
\end{tabularx}

\paragraph{1785} \intro{Mr.\ Kopf, on the first day of music theory class}: ``I don't have time to play you all the music in the world.''

\paragraph{1792} \intro{Tess:} ``I think I've learned three things from English class:
\begin{enumerate}
  \setlength{\itemsep}{-3pt}
  \item People are idiots.
  \item Don't cheat on one another.
  \item Don't think you're God.''
\end{enumerate}

\paragraph{1804} \intro{Overheard in the hall}: ``We have to do all the parts of speech, like personification and irony.''

\paragraph{1808} \intro{Seen at the bottom of a webpage}: ``Last modified May 16, 1912.''

\paragraph{1816} \intro{Cheryl}: ``I had a blond senior moment.''

\paragraph{1818} \intro{A classmate, about the translation of the novel we were reading}: ``So if [Isabel Allende] writes everything on the computer now, why doesn't she just put it in Google Translate?''

\paragraph{1847} \intro{Me, in a dream}: ``Pssh, that's all right. I wash my face in Tennessee.''

\paragraph{1856} We have a standing joke in orchestra (based largely on reality) that Mr.\ Kopf never seems to hear what Tess says. On this occasion, he gives us a puzzle, and people, including her, are shouting out possible answers. After a moment:\\

\noindent \begin{tabularx}{\textwidth}{r X}
  \st{Mr.\ Kopf} & ``I haven't heard the right answer yet\ldots "\\
  \st{Me} & \emph{(looking over my shoulder at Tess)} ``Maybe you're right then!"
\end{tabularx}

\paragraph{1891} \intro{A classmate, contradicting me:} ``No, Mozart was Swedish.''

\paragraph{1951} \intro{I was proofreading a classmate's essay and came across this sentence}: ``Overall, Blake is an interesting poem.''

\paragraph{1966} \intro{Mr.\ Kopf, on the first day after Tess had had vocal surgery and had very strict rules about when she was allowed to talk}: ``Ah, Tess is being quiet today! What a wonderful thing!''

\paragraph{1972} Mr.\ Kopf told us a story today about a time he quadruple-booked himself. He had two concerts, a dress rehearsal, and another rehearsal scheduled all on the same night. By the time he could get himself to do anything about it (he delayed because he was embarrassed and worried about it), he didn't have time to work something out. So he excused himself from the rehearsal, saying he had something important to do (not entirely untrue!), and he called someone at the dress rehearsal and told them he was deathly ill and couldn't make it.

Of course, this still left two concerts. So he went to the first one and played for a while, then ``got sick," drove to the other one, and just got there a little bit late! Nobody ever figured it out because the venues were far enough apart that nobody knew him in both places.

\paragraph{1980} \intro{Asked of my mother and me}: ``How long have you been married?''

\paragraph{1994} \intro{Mr.\ Kopf}: ``If I'm wrong, I might be wrong.''

\chapter{2013}
\paragraph{2026} \intro{Mr.\ Kopf}: ``It wouldn't be wrong, but it wouldn't be right.''

\paragraph{2032} \intro{A friend, to me, incredulously}: ``You have a to-do list?''

\paragraph{2035} \intro{In the changelog for a flashlight app, which basically makes the screen white and turns on the flash}: ``Improved performance and stability.''

\paragraph{2059} \intro{Google Voice voicemail transcription of ``Luther College''}: ``He works at the loser college.''

\paragraph{2068} Jesus and the devil get in an argument about who has the better computer skills. To settle the argument, they get an arbiter to arrange an hour-long contest during which they have to complete a number of tasks. Five minutes before the end of the allotted time, the power goes out momentarily and the computers die. The devil immediately jumps up from his seat and starts swearing, but Jesus is unperturbed. The devil looks at Jesus and asks angrily, ``Why is he so calm?''

``Well,'' the arbiter replies, ``Jesus saves.''

\paragraph{2072} \intro{A classmate, presenting in English class}: ``Misunderstatement.''

\paragraph{2110} \intro{A contributor to a wiki, after being told title case was inappropriate for the body text of articles}: ``In case you Can't tell I'm trying to Not Capitalize every word.''

\paragraph{2124} \intro{Overheard on the bus}: \\

\noindent \begin{tabularx}{\textwidth}{r X}
  \st{Girl 1} & ``I can't pee my pants because I'm wearing a dress.'' \\
  \st{Girl 2} & ``Thanks for sharing.''
\end{tabularx}

\paragraph{2236} \intro{Overheard on the bus}: ``Are you kidding me? I'd just rather \emph{not} die.''

\paragraph{2290} \intro{On the Amazon page for a blank journal}: ``Publisher: learn how customers can search inside this book.''

\paragraph{2300} \intro{Overexcited girl with loud, high-pitched voice, passing me in the hall}: ``I never wear real pants. Pants are too much work!''

\paragraph{2322} \intro{Overheard in the hall}: ``That's our senior prank. Let's kill Melissa as our senior prank!''

\paragraph{2326} \intro{Automated phone system}: ``Please enter your phone number, one digit at a time.''

Nah, I'll just go for pressing them all at once.

\paragraph{2329} Marco was looking at my yellow \textsc{lsm} shirt today. I knew something was funny as soon as he said, ``Do you like Lutheran Summer Music?'' It was just something about the way he said it.

Somewhat quizzically, I said, ``Yeah?"

``Is that like its own genre or something?''

\paragraph{2344} \intro{Overheard in the hall}: ``I will give you the Nobel Prize if you just shut the fuck up!''

\paragraph{2346} \intro{While practicing a speech}:\\

\noindent \begin{tabularx}{\textwidth}{r X}
  \st{Me} & ``Someone give me a country\ldots any country.''\\
  \st{Student} & ``Africa.''
\end{tabularx}

\paragraph{2348} \intro{Another student in my English class, to me}: ``What's the plural of `analysis'? `Analysises'?''

\paragraph{2394} \intro{Subject line of a classified ad}: ``Offer: huge bag of teen girls''

(Fortunately, the body\footnote{Pun absolutely intended.} talked about \emph{clothing} for teen girls.)

\paragraph{2440} \intro{A technician ``calling from Windows,'' describing the ``errors'' that my computer supposedly had}: ``It's eating your computer, sir, day by day.''

\paragraph{2474} \intro{My dad}: ``You can lead a fly to freedom, but you can't make it\ldots free. Deep.''

\paragraph{2519} \intro{On a package of plastic screen protectors, underneath the picture of the product in use}: ``Phone sold separately.''

\paragraph{2534} \intro{Name of a wireless network}: ``Screaming Toilet''

\paragraph{2539} \intro{A woman who claimed to be a locksmith}: ``This is a deadbolt, so you can't pick it.''

I should also note that her primary tool was a DeWalt cordless drill.

\paragraph{2604} \intro{In the terms and conditions for a Discover credit card}: ``You will not earn cashback bonus on illegal transactions.''

\paragraph{2610} \intro{Written on the whiteboard at the \textsc{it} helpdesk}: ``Authentification error''

\paragraph{2627} \intro{Dr.\ Aspaas, jogging to the back of the Chapel to show us how to process}: ``Look! I'm running in church!''

\paragraph{2708} \intro{End of a phishing message}: ``It is risk free and legal.''

\paragraph{2731} \intro{A classmate for whom English is a second language, in a report we were writing together}: ``This aberration is quite unique in the data trend.''

The worst part was that I could only change so much of it because he kept talking about how much he liked the word ``aberration'' and how it was perfect here (it was completely incorrect).

\paragraph{2767} \intro{Dr.\ Aspaas}: ``This is like slices of meat. It needs to be like rockets in the sky.''

\paragraph{2791}
On Choral Day, we were singing ``E'en so, Lord Jesus, Quickly Come.'' The tenors have this beautiful pianissimo suspension into a dominant seventh at the end. The guy next to me, though, kept dropping when the other parts moved and singing the resolved note right away, which sounded absolutely awful, but apparently he didn't notice this. I explained his mistake to him at least once, maybe even twice. Finally, after hearing it about five times, I leaned and directed my voice in his direction as politely but obviously as possible and sang it forte (no doubt some other people wondered what that moron was doing singing this line forte).

He figured it out, though. A couple of seconds after the piece finished, he said, quote, ``Ohhh.''

\paragraph{2792} \intro{Overheard in the hall, one young woman to another}: ``I'm the one that spills the tea in this relationship.''

\paragraph{2850} \intro{Sigrid, after someone played the starting pitch during rehearsal}: ``Play that again\ldots Talk to me about that pitch pipe.''

\paragraph{2950} \intro{Printed at the end of a Latin final exam, next to the honor code}: ``Gloria in excelsis Deo!''

\paragraph{2977} \intro{Overheard in the waiting room at the dentist}: ``I'd rather have a baby than get my teeth cleaned.''\thinspace\footnote{For those of you wondering, it was a woman, and she said she had had children.}

\paragraph{3005} \intro{Warning on a kitchen fire extinguisher}: ``Do not discharge at people's faces.''

\chapter{2014}
\paragraph{3064} \intro{Me, to the editors of my copy of the \emph{Aeneid}}: ``Thanks for putting spoilers in the glossary!''

\paragraph{3066} I was at Viking rehearsal last Tuesday, and as I was coming in and putting my stuff somewhere, someone that I didn't know asked me, ``Did you find those gloves?'' I was not convinced he was talking to me until he had said it several times and started to get slightly annoyed, because it sounded like he was talking to someone who'd previously lost his gloves (which I had not). I then was still loath to assume he was talking about the ones I had with me, because not only were they halfway into my coat pocket and not being used or touched in any way and were thus hard to correctly describe as ``those,'' they were also mittens. After about a minute of confused discussion, I finally figured out what he meant: apparently he had recently lost some gloves that looked similar to mine and was wondering if I had picked them up and was now using them. Talk about a long shot\ldots

\paragraph{3067} \intro{At the bottom of a tax document}: ``For Paperwork Reduction Act Notice, see the separate instructions.''

\paragraph{3107} \intro{Overheard on campus}: ``You know what would be really awesome? Two monocles.''

\paragraph{3138} \intro{Me, at the beginning of my first semester}: ``Call me naïve, but I was expecting there to be at least some \emph{music} in the \emph{music} folder we checked out from the \emph{music} library.''

\paragraph{3151} \intro{Annika, when I told her I needed to leave because I had a shift at the \textsc{it} helpdesk}: ``Go sit and be helpful.''

\paragraph{3164} \intro{A comment thread on Slashdot}:\\

\noindent \begin{tabularx}{\textwidth}{r X}
  \st{Person 1} & ``It's about giving freedom to the code.'' \\
  \st{Person 2} & ``I dunno about you, but I've never had any code I've written pass a Turing test and then demand emancipation.''
\end{tabularx}

\paragraph{3203} \intro{Anna-Christina}: ``Jesus, mother of God!''

\paragraph{3216} Annika, Joseph, someone else I don't remember, and I were having a conversation at one end of a long table at lunch today. Right after Annika said something, someone several chairs down who had not been part of the conversation leaned over and said, ``Did you just say your mom thinks you're a prostitute?'' 

(She hadn't.)

\paragraph{3276} Professor Olaf told us he once ended up working as a waiter in a five-star restaurant in France, without having any idea that was what he was getting into. He'd visited a job recruiter, who'd asked if he had any restaurant experience, but never bothered to ask what kind. His previous experience was delivering pizzas.

\paragraph{3339} \intro{Dr.\ Aspaas}: ``I like music.''

\paragraph{3389} Professor Reece told us a story about Dr.\ Felland, the university's first classics professor, known as the ``absentminded professor.'' One day he showed up to speak in Chapel wearing a suit but no tie. (At the time, there was a pretty strict dress code and this was a big deal -- and it would look a bit silly even today.) People must have been looking at him funny or something, because he noticed, and without missing a beat, he announced, ``The opening hymn for today will be `Blest Be the Tie that Binds'.'' Then he ran back to his office during the hymn and got a tie.

\paragraph{3435} \intro{Overheard on campus, in response to someone who said he couldn't come to Chapel today}: ``Is homework more important than God? Huh? Huh?''

\paragraph{3519} It was the day of our Latin midterm. Professor Reece had stepped out of the room for a moment, but the time for class to start had come and the exams were sitting on the table, so we decided we would start. There was a large cookie tin on top of them.\\

\noindent \begin{tabularx}{\textwidth}{r X}
\st{Student} & ``Are these cookies?'' \\
\st{Me} & \emph{(sarcastically)} ``No, it's a bomb, so it blows up in your face when you try to get cookies.''\\
\st{Student 2} & ``That would suck. Right before spring break and all.''\\
\st{Student 3} & ``That's the only time a bomb would suck?''
\end{tabularx}

\paragraph{3564} \intro{A conductor on the Empire Builder}: ``We do not foresee any emergencies today.''

\paragraph{3586} \intro{Dr.\ Aspaas}: ``You guys are feisty today! I like it! \ldots Shuddup!''

\paragraph{3658} \intro{A caller to the \textsc{it} helpdesk, explaining her call}: ``I'm having a problem with my computer.''

\paragraph{3680} \intro{Another Slashdot comment, about what would happen if the battery in a smart gun died}: ``[Smart guns] should just do something analogous to what smoke detectors do. Like every few minutes when the battery gets low, automatically fire off a round.''

\paragraph{3730} Someone fried our dorm's microwave by blindly following directions from the Web and microwaving a potato for 20 minutes. (Add that to the mental ``things never to do'' list\ldots) That was an expensive, new microwave, and nobody in the dorm can heat anything up now.

This is also the second ruined microwave this semester. Although the first one was a crappy little thing and it broke when someone cracked the carousel into two pieces and we decided it wasn't worth repairing, so this time was definitely worse.

I'm not sure whether to be more annoyed at the fact that somebody stupidly put a potato in the microwave for 20 minutes unattended or the fact that a two-hundred-dollar piece of equipment can be ruined by cooking a potato for too long.

\paragraph{3802} \intro{Advertised on the side of a \textsc{wm} truck}: ``Our landfills provide over 17,$\:\!$000 acres of wildlife habitat.''
% that is idiotic kerning...

\paragraph{3891} \intro{Amazon review of a ballpoint pen}: ``This pen is easy to use.''

\paragraph{3914} An Alsatian dog went to a telegram office, took out a blank form, and wrote, “Woof. Woof. Woof. Woof. Woof. Woof. Woof. Woof. Woof.”

The clerk examined the paper and politely told the dog: “There are only nine words here. You could send another \emph{Woof} for the same price.”

“But,” the dog replied, “that would make no sense at all.”
\paragraph{3984} \intro{An automated phone system, administering a survey}: ``Your response was `yes'. If this is correct, say `yes'.''

\paragraph{4029} \intro{Lifehacker comment on an article about how to drive so as to get stopped at fewer red lights}: ``I have found that if you need to do anything important at a stoplight, you will never get a red light. So, always have something that needs to be done at that next stoplight when you're driving.''

\paragraph{4101} \intro{Bryan, on our first night in Mellby}: ``I think we've already talked more than my roommate and I did all last year.''

\paragraph{4127} \intro{Dr.\ May, on how to have success in his class}: ``Come to class, do your work, say your prayers. That third one's essential.''

\paragraph{4129} \intro{Dr.\ May}: ``Verbs have moods, right? Sometimes they're happy, sometimes they're sad.''

\paragraph{4165} \intro{Dr.\ Aspaas}: ``[That was] a bit like audible Whack-a-Mole.''

\paragraph{4170} \intro{Dr.\ May}: ``They sent some people to assassinate Cicero, but he knew about it. He didn't let them in.''

\paragraph{4186} \intro{Joseph, about Professor Brunelle}: ``So, he's presenting a paper on pillows in Poland. I'm not sure if he planned that alliteration or not.''

\paragraph{4229} \intro{St.\ Olaf Account Services, on trying to change a password}: ``Please select a password that does not have a suffix commonly used for women's names.''

\paragraph{4231} \intro{Dr.\ Hodel}: ``[Rubato is] where you steal from one note and give it to the next. It's like Robin Hood, only with rhythm.''

\paragraph{4237} \intro{Dr.\ May}: ``Wow, you guys are smart. You must have gone to college. Or\ldots\\ you're \emph{in} college.''

\paragraph{4247} At the beginning of Latin class one day:\\

\noindent \begin{tabularx}{\textwidth}{r X}
\st{Dr.\ May} & \emph{(coming in the door)} ``You're sitting in the dark. Does that describe your mental state?''\\
\st{Student} & ``Yes.''\\
\st{Dr.\ May} & \emph{(flipping the light switch)} ``Fiat lux.''\thinspace\footnotemark
\end{tabularx}
\footnotetext{\emph{Let there be light} (Gen. 1:3)}

\paragraph{4284} Dr.\ May told us that he was leading an interim trip abroad once, and one of the assignments was to keep a journal while they were there. He had one student turn in his copy with about five pages paperclipped together and a sticky note reading, ``Dr.\ May, don't read this part.'' Of course, he did. It was about some romantic encounter he'd had with this Greek girl, but it wasn't anything particularly bad.

Seriously, though, did the guy have no other paper? It wasn't like he didn't know he was going to be asked to turn it in and have his professor read it.

\paragraph{4290} \intro{Dr.\ May}: ``People ask me, what's it like to be dean? I tell them, being dean of the college is like being a psychologist or a priest: All these people come in and tell you things you wish you never heard, and then you can't tell anyone about it.''

\paragraph{4330} \intro{Part of a Lifehacker comment}: ``Activate your hazards, or, as my daughter calls them, your park-anywhere lights.''

\paragraph{4335} \intro{Overheard from the next dorm room, the end of a phone conversation}: ``All right, I have to go and build my resume.''

\paragraph{4344} \intro{Dr.\ May}: ``[The exam] won't be impious, but it could be nefarious.''

\paragraph{4369} \intro{Dr.\ Aspaas}: ``It's not rocket surgery.''

\paragraph{4388} Dr.\ May told us a story today. He was teaching an introductory Latin course, and one day he wrote \emph{Carpe diem} on the board and asked if someone could tell him what it meant. Some guy raised his hand and said confidently, ``Fish of God.''\thinspace\footnote{\emph{Pisces Dei}, perhaps?}

\paragraph{4407} \intro{Overheard in the caf line}: ``Is it bad if I have \emph{two} bananas?''

\paragraph{4434} \intro{Caller to the \textsc{it} helpdesk}: ``I just need to talk to someone who knows a decent amount about things.''

\paragraph{4443} \intro{Dr.\ May}: ``Any problems? \ldots Latin problems. I know you have problems.''

\paragraph{4484} \intro{Dr.\ Aspaas}: ``I have always been trying to reform Choral Day. But it's a bit like trying to steer an aircraft carrier with a tongue depressor.''

\paragraph{4507} \intro{Dr.\ Hodel}: ``Musicians have to count.''

\paragraph{4516} One piece we were doing in Chapel Choir required two different ``choirs'' with their own parts. The day after we started it, some people had forgotten which one they'd been assigned to.\\

\noindent \begin{tabularx}{\textwidth}{r X}
\st{Person 1} & ``Raise your hand if you don't know which choir you're in.'' \\
\st{Person 2} & \emph{(to the people who have raised their hands)} ``Chapel Choir.''\\
\end{tabularx}

\paragraph{4555} \intro{Dr.\ Aspaas}: ``Don't be a balloon. Eeeeeeeeeeee!''

\paragraph{4584} \intro{Dr.\ Aspaas}: ``It's like walking across the Grand Canyon on dental floss.''

\paragraph{4596} \intro{Sigrid}: ``I like wobble!''

\paragraph{4606} \intro{Dr.\ Bobb}: ``It sounds a bit like stepping on a cat's tail.''
\paragraph{4672} \intro{Someone at a church I was visiting, offering extra tickets}: ``I have two extra tickets to the Minnesota Sympathy."

\paragraph{4695} \intro{While walking to dinner after choir rehearsal, I overhear the following between two Chapelites, sung to ``Love Has Come"}:\\

\noindent \begin{tabularx}{\textwidth}{r X}
\st{Girl 1} & ``Help! Help! I am such a weirdo.'' \\
\st{Girl 2} & ``Help! Help! I'll punch you in the face!'' \\
\end{tabularx}

\paragraph{4720} \intro{Jaynee, before Christmas Festival, trying to express the number of performances remaining}: ``That's how it works. Two minus two equals two."

\paragraph{4721} I had to stop to tie my shoe in the vestibule of Skoglund on my way out of Christmas Festival on Friday night, and there were two members of Manitou standing there. The one girl says to the other: ``I'm so emotional tonight. I don't know why."

\paragraph{4793} \intro{Dr.\ May}: ``It's my favorite construction. It's the double dative! I have to announce this to the masses.'' \emph{(opens door, pokes head out)}\enspace ``IT'S THE \mbox{DOUBLE} $\:\!$DATIVE!''\enspace \emph{(slams door)}

\backmatter
\emph{In the original version, there was a glossary here listing full names and descriptions of each person mentioned. In the interests of protecting people's privacy, I have not included it in this publicly available version. Thanks to an anonymous person (formerly listed here) for pointing this out as a problem.}

\end{document}
